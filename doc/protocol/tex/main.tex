\documentclass[a4paper, 11pt]{article}

\usepackage[margin=1in]{geometry}

\usepackage{hyperref}
\usepackage{graphicx}
\usepackage{graphics}
\usepackage{verbatim}
\usepackage{listings}
\usepackage{color}

\begin{document}

\title{The BANO protocol}
\author{BANO team}
\date{}

\maketitle

%% \newpage
%% \tableofcontents
%% \addtocontents{toc}{\protect\setcounter{tocdepth}{1}}


\clearpage
\section{Protocol overview}

\paragraph{}
The BANO protocol has been designed to implement a wireless network where a
resourceful device known as the base interacts with smaller devices known as
the nodes. Applications includes domotics.

\paragraph{}
In a typical architecture, the base centralizes node states. The base reports
node information to the user, and schedules appropriate actions based on the
user configuration.

\paragraph{}
To ease conceptual understanding, this document focuses on the simple case where
base and node roles are clearly distinct. However, there is no limitation that
prevents a node to act as a base, and a base to act as a node. Also, the protocol
does not limit the base count.

\paragraph{}
The BANO protocol is used between bases and nodes for accessing key value pairs
using 2 basic operations:
\begin{itemize}
\item \textit{SET}: used to set a value given a specific key,
\item \textit{GET}: used to get a value given a specific key.
\end{itemize}

\paragraph{}
These 2 operations are used to implement the following:
\begin{itemize}
\item \textit{synchronous SET} with optionnal acknowledgement: a base can set a
node value by sending a \textit{SET} message. If required, an acknowledgement can
be sent back as a flagged message,
\item \textit{synchronous GET}: a base can get a node value by sending a
\textit{GET} message, and wait for the corresponding reply,
\item \textit{asynchronous SET}: a base can get a node value by listening for
\textit{SET} messages coming from nodes.
\end{itemize}


\subsection{Design considerations}

\subsubsection{Focus on common case}
\paragraph{}
The protocol is designed with common case in mind. Any feature that is not
mandatory should not impact its design. For instance, security is a feature
that would impact message format if made mandatory. Another example is the
lack of message routing support.

\subsubsection{Node simplicity}
\paragraph{}
Nodes resource requirements should be as low as possible, enabling low cost
8 bits microcontroller based configurations. On the contrary, the base is
considered resourceful. This should be used to lower the node requirements.

\subsubsection{Low power consumption}
\paragraph{}
\textbf{TODO}
%% TODO: node mode (passive, listen only, time ...)

\subsubsection{Low memory footprint}
\paragraph{}
\textbf{TODO}


\subsection{Transport layer}
\paragraph{}
BANO aims at being independent from the physical layer and limits the
constraints put on the layer used to transport messages. It requires:
\begin{itemize}
\item packet ordering is not required,
\item packet acknowledgement is not required,
\item a fixed size packet based transport layer,
\item if provided by hardware, addressing must be at least 4 bytes.
Otherwise, addressing is implement in software and the payload size must be
increased by 4 bytes,
\item if implemented by hardware, the CRC must be at least 2 bytes. Otherwise,
the CRC can be implemented in software,
\item data payload size must be at least 16 bytes, and depends on what is
implemented in software.
\end{itemize}

\paragraph{}
The physical layer can work in all the commonly used wireless ranges (433, 868,
915 MHz and 2.4 GHz). 2 chipsets are considered:
\begin{itemize}
\item NRF905
\item NRF24L01P
\end{itemize}


\subsection{Addressing}
\paragraph{}
Network nodes are addressed using 32 bits identifiers. Addresses are randomly
generated at programming time, along with a random 32 bits seed. At any time,
the base can detect address collision by checking the uniqueness of the
addr,seed pair. To do so, it sends a message to all the nodes that it has
currently discovered:
%% msg.daddr = node_addr
%% msg.hdr.op = BANO_OP_GET
%% msg.hdr.flags = 0
%% msg.hdr.saddr = base_addr
%% msg.u.set.key = BANO_KEY_ADDR

\paragraph{}
Nodes reply with the following message:
%% msg.daddr = base_addr
%% msg.hdr.op = BANO_OP_SET
%% msg.hdr.flags = BANO_FLAG_REPLY
%% msg.hdr.saddr = node_addr
%% msg.u.set.key = BANO_KEY_ADDR
%% msg.u.set.val = node_seed

\paragraph{}
If there are 2 random ids for one address, a collision is detected. The base
resolves the conflict and sends new addresses to the node:
%% msg.daddr = node_previous_addr
%% msg.hdr.op = BANO_OP_SET
%% msg.hdr.flags = 0
%% msg.hdr.saddr = node_new_addr
%% msg.u.set.key = BANO_KEY_ADDR
%% msg.u.set.val = node_seed

\paragraph{}
By default, the base uses the BANO_DEFAULT_BASE_ADDR address.


\subsection{Messaging}
\paragraph{}
in RX mode, a node receive logic consumes power. to reduce power consumption and
avoid being, a node can poll the requests by sending a request:
BANO_OP_GET, BANO_KEY_REQ, node_addr
the base must then send requests for this node in a short time lapse, and the
node will then return to reduced power mode at completion.
also, note the polling scheme prevents a malicious sender to
keep transmitting to a node in order to reduce its battery life.

%% TODO
%% [ on message forwarding ]
%% Some physical setup may require the use of repeater.

\paragraph{}
TODO: message integrity

\paragraph{}
TODO: message acknowledgement


\subsection{Security}
\paragraph{}
Implementing security adds complexity not required in all cases and goes against
some other BANO protocol design choices (stateless, common case, short message
size). However, it is undeniable that there are node messages that require some
or all of the following security features:
. hiding message contents
. preventing replay attack
. preventing brute force
. preventing tampering attacks (mim ...)
. detecting flood
. node DOS


[[ message contents hiding ]]

Symetric block ciphers can be used to hide message contents. The cipher
block size should comply with the BANO message size, ie. 16 bytes. Thus,
a 128 bits cipher is chosen (AES 128). Also, encryption is enabled for
all the messages of a given node. If this is a limitation, a node key
can allow the node to switch switch the encryption mode. In this case,
and because of statelessness, the encryption mode toggling should be
sent in 2 encrypted and clear versions.

Ciphers rely on a secret key shared between the base and the node. There
is no support for sharing the key built in the protocol. In a typical
situation, the node is programmed with a random key that is made known
to the base using the configuration interface.


[[ preventing replay attack ]]

The usual way to prevent replay attack is for the node message to convey
a seed. This seed must not be known by the attacker at the time the message
is generated.

example: shutting down an alarm
An attacker may replay a previously capture message used to
unlock an alarm, even if encrypted. If the message contains
a time dependent information that can be checked by the base,
then any previously generated message will no longer be valid.
The time dependent information should be generated randomly
in a fashion not known by the attacker. For instance, a seed
can be generated locally by the base and sent encrypted to
the node.

[[ preventing brute force ]]

[[ preventing tampering attacks (mim ...) ]]

a hash of the message payload is included in the message. the message
is then encrypted. Thus, a modification


[[ detecting node flood ]]
flood can be used or to prevent a node from speaking with
the base.


. security seed broadcast
-> the master (node?) broadcasts a seed at regular interval his security
seed is used in the encryption token scheme
-> if the node broadcast, it must be able to generate it, which means random
number generation caps
-> if the master broadcast it, the node has to store it

=> any failed attempt to check the security token should renegotiate the
node security context


%% TODO
%% [ on the assumption base is always present before nodes ]

%% A scenario where the user powers a node before powering the
%% base is largely possible. Handshaking .

%% Also, the base could be powered

%% Also, completely passive nodes (ie. alarm) does not anounce
%% themselves when powered up.

%% Thus, the base is not assumed to be present when a node is
%% added to the network.

%% (or base cannot assume base was present at the time
%% they do). Thus, there must be a mechanism to convey every
%% required information in one message.

%% Designing the protocol this way will increase



\section{Protocol implementation}
\paragraph{}
The protocol uses little endianess for any field wider than one byte.
\paragraph{}
\textit{keys} are 16 bits identifiers. Each node exposes a set of predefined keys
and a set of specific keys.
\paragraph{}
\textit{value} are 32 bits wide. The protocol does not specify a format for this
field. For instance, it can be used to transmit integer or real values. The base
uses the node description file to interpret the value of a specific key.

\subsection{Message format}
16 bytes
-- common
op:2
flags:6
saddr:32
aux: 24
-- per message
key:16
val:32
checksum: 16

%%
%% TODO: review existing protocols


\end{document}
