\documentclass[a4paper, 11pt]{article}

\usepackage[margin=1in]{geometry}

\usepackage{hyperref}
\usepackage{graphicx}
\usepackage{graphics}
\usepackage{verbatim}
\usepackage{listings}
\usepackage{color}

\begin{document}

\title{BANO node software development kit}
\author{texane@gmail.com}
\date{}

\maketitle

%% \newpage
%% \tableofcontents
%% \addtocontents{toc}{\protect\setcounter{tocdepth}{1}}


\clearpage
\section{Overview}

\paragraph{}
This documents the BANO node software development kit, including
the build environment and programming interface. It is primarly
intended for node developers.

\subsection{Installing SDK and dependencies}

\paragraph{}
The BANO SDK depends on the AVR GNU toolchain being installed on
the system and accessible using the PATH environment variable.
On a LINUX DEBIAN system, this toolchain is installed using:
\begin{scriptsize}
\begin{verbatim}
$> aptitude install gcc-avr
\end{verbatim}
\end{scriptsize}

\paragraph{}
Also, the AVRDUDE tool is used to upload the binary file in the
device flash:
\begin{scriptsize}
\begin{verbatim}
$> aptitude install avrdude
\end{verbatim}
\end{scriptsize}

\paragraph{}
Note that this tools are shipped with the ARDUINO environment.

\paragraph{}
Retrieving the BANO SDK requires cloning 3 repositories:
\begin{itemize}
\item the BANO SDK itself, which contains all the files required
to develop node and a base applications,
\item the NRF SDK, which provides support for NRF905 and NRF24L01P
wireless chipsets from Nordic Semiconductor,
\item the AVR CRYPTO library, which is used for message encryption.
\end{itemize}

\paragraph{}
It is recommended to install them at the same level:
\begin{scriptsize}
\begin{verbatim}
$> cd ~/repo
$> git clone https://github.com/texane/bano.git
$> git clone https://github.com/texane/nrf.git
# only for message encryption
$> mkdir ~/repo/crypto-lib
$> cd ~/repo/crypto-lib
$> svn co http://das-labor.org/svn/microcontroller-2/crypto-lib .
\end{verbatim}
\end{scriptsize}

\paragraph{}
Also, a node example example can be retrieved, built and uploaded
using:
\begin{scriptsize}
\begin{verbatim}
$> git clone https://github.com/texane/bano_led.git
$> cd bano_led
$> BANO_DIR=~/repo/bano make
$> BANO_DIR=~/repo/bano make upload
\end{verbatim}
\end{scriptsize}

\clearpage
\section{Build environment}

\subsection{Makefile variables}
\paragraph{}
BANO allows the developer to interact with the build system by
defining or overriding default variables in the node makefile:
\begin{itemize}
\item BANO\_DIR: the BANO top directory. If the previous section
instructions were followed, the BANO\_DIR variable would be set
to \$(HOME)/repo/bano. This variable is mandatory,
\item NRF\_DIR: the NRF top directory. default to
\$(BANO\_DIR)/../nrf,
\item CRYPTOLIB\_DIR: the CRYPTO-LIB top directory. default to
\$(BANO\_DIR)/../crypto-lib,
\item BANO\_BOARD: the node board. Currently, only minipro\_3v3
is supported. This variable is mandatory,
\item NRF\_DIR: the NRF top directory. By default, the SDK
assumes that the NRF directory is located in the BANO parent
directory, as recommended in the previous section,
\item BANO\_C\_FILES: the list of C source files. This list
can not be empty,
\item BANO\_S\_FILES: an optional list of assembler source files.
Names must end with big S suffixes (ie. main.S). GCC is used as
the assembler allowing preprocessor directives. This list can be
empty,
\item BANO\_NODL\_ID: the NODL identifier as an hexadecimal
32 bits unsigned integer. If not defined, 0 is used,
\item BANO\_NODE\_ADDR: the node address as an hexadecimal 32
bits unsigned integer. If not defined, a random value is
generated by the build system,
\item BANO\_NODE\_SEED: the node seed as an hexadecimal 32
bits unsigned integer. If not defined, a random value is
generated by the build system,
\item BANO\_CIPHER\_ALG: if defined, the 128 bits block cipher
algorithm to use. Currently supported are: none, xtea and aes,
\item BANO\_CIPHER\_KEY: if BANO\_CIPHER\_ALG is defined, this
variable contains the key used to encrypt and decrypt messages.
It is represented as a comma separated string of 16 8 bits
unsigned integers in hexadecimal. If not defined, a random value
is generated by the build system.
\end{itemize}

\subsection{Makefile rules}
\paragraph{}
The makefile provides the following rules:
\begin{itemize}
\item all: also the default rule. It produces the final image
that can then be uploaded,
\item upload: upload to the board,
\item clean: clean all the temporary files.
\end{itemize}

\subsection{Example makefile}
\begin{scriptsize}
\begin{verbatim}
# node built for the arduino minipro 3.3v board
# it assumes that BANO_DIR is defined by the environment
# it consists of a single C file main.c
# the node address is statically defined

BANO_DIR := $(HOME)/repo/bano
BANO_BOARD := minipro_3v3
BANO_NODE_ADDR := 0x5c5f8548
BANO_C_FILES := main.c

include $(BANO_DIR)/build/node/top.mk
\end{verbatim}
\end{scriptsize}


\clearpage
\section{Programming reference}

\subsection{Overview}
\paragraph{}
BANO offers a runtime and the corresponding programming interface
that abstracts a node application logic from low level details
such as hardware architecture and protocol implementation. The
interface is mainly descriptive and event based: the developer
first initializes node related information. The BANO runtime then
calls application handlers whenever appropriate: network messages
reception, timers, hardware related interrupts ...

\subsection{Files}
node/bano\_node.h: function and type declarations.

\subsection{Types}
\begin{scriptsize}
\begin{verbatim}
typedef struct
{
  /* 100 milliseconds units, max 10736 */
  uint16_t timer_100ms;

  /* waking event mask */
#define BANO_WAKE_NONE 0
#define BANO_WAKE_TIMER (1 << 0)
#define BANO_WAKE_MSG (1 << 1)
#define BANO_WAKE_POLL (1 << 2)
#define BANO_WAKE_PCINT (1 << 3)
  uint8_t wake_mask;

  /* module disabling mask */
#define BANO_DISABLE_ADC (1 << 0)
#define BANO_DISABLE_WDT (1 << 1)
#define BANO_DISABLE_CMP (1 << 2)
#define BANO_DISABLE_USART (1 << 3)
#define BANO_DISABLE_NONE 0x00
#define BANO_DISABLE_ALL 0xff
  uint8_t disable_mask;

  uint32_t pcint_mask;

} bano_info_t;

static const bano_info_t bano_info_default =
{
  .wake_mask = BANO_WAKE_NONE,
  .disable_mask = BANO_DISABLE_ALL
};
\end{verbatim}
\end{scriptsize}


\subsection{Functions}
\begin{scriptsize}
\begin{verbatim}
/* exported by the runtime */
uint8_t bano_init(const bano_info_t*);
uint8_t bano_fini(void);
uint8_t bano_send_set(uint16_t, uint32_t);
uint8_t bano_wait_event(bano_msg_t*);
uint8_t bano_loop(void);
\end{verbatim}
\end{scriptsize}

\begin{scriptsize}
\begin{verbatim}
/* implemented by the application */
extern uint8_t bano_set_handler(uint16_t, uint32_t);
extern uint8_t bano_get_handler(uint16_t, uint32_t*);
extern uint8_t bano_timer_handler(void);
extern uint8_t bano_pcint_handler(void);
\end{verbatim}
\end{scriptsize}

\clearpage
\section{Examples}

\subsection{Enable disable a LED on BANO messages}
\begin{tiny}
\begin{verbatim}
#include <stdint.h>
#include <avr/io.h>
#include "bano/src/node/bano_node.h"

/* led routines */

#define LED_DDR DDRB
#define LED_PORT PORTB
#define LED_MASK (1 << 1)

static void led_set_high(void)
{
  LED_DDR |= LED_MASK;
  LED_PORT |= LED_MASK;
}

static void led_set_low(void)
{
  LED_DDR |= LED_MASK;
  LED_PORT &= ~LED_MASK;
}


/* event handlers */

#define LED_KEY 0x0000

static uint8_t led_value = 0;

uint8_t bano_get_handler(uint16_t key, uint32_t* val)
{
  /* called by the runtime on GET messages */

  if (key != LED_KEY) return (uint8_t)-1;
  *val = led_value;
  return 0;
}

uint8_t bano_set_handler(uint16_t key, uint32_t val)
{
  /* called by the runtime on SET messages */

  if (key != LED_KEY) return (uint8_t)-1;
  led_value = val;
  if (led_value == 0) led_set_low();
  else led_set_high();
  return 0;
}

/* unused */
uint8_t bano_pcint_handler(void) { return (uint8_t)-1; }
uint8_t bano_timer_handler(void) { return (uint8_t)-1; }

int main(void)
{
  /* initialize the runtime and loop forever */

  bano_info_t info;

  info = bano_info_default;
  info.wake_mask |= BANO_WAKE_MSG;
  bano_init(&info);

  bano_loop();

  bano_fini();

  return 0;
}

\end{verbatim}
\end{tiny}


\clearpage
\subsection{Send periodic messages}
\begin{tiny}
\begin{verbatim}
#include <stdint.h>
#include <avr/io.h>
#include "bano/src/node/bano_node.h"

uint8_t bano_timer_handler(void)
{
  /* called every 10 seconds */

  bano_send_set(0x002a, 0xdeadbeef);
  return 0;
}

/* unused */
uint8_t bano_get_handler(uint16_t key, uint32_t* val) { return (uint8_t)-1; }
uint8_t bano_set_handler(uint16_t key, uint32_t val) { return (uint8_t)-1; }
uint8_t bano_pcint_handler(void) { return (uint8_t)-1; }

int main(void)
{
  bano_info_t info;

  info = bano_info_default;

  info.wake_mask |= BANO_WAKE_TIMER;
  info.timer_100ms = 100;

  bano_init(&info);

  bano_loop();

  bano_fini();

  return 0;
}
\end{verbatim}
\end{tiny}


\clearpage
\subsection{Send message when GPIO changes}
\begin{tiny}
\begin{verbatim}
#include <stdint.h>
#include <avr/io.h>
#include "bano/src/node/bano_node.h"

#define GPIO_KEY 0x0000

#define GPIO_DDR DDRD
#define GPIO_PIN PIND
#define GPIO_MASK (1 << 3)

uint8_t bano_pcint_handler(void)
{
  bano_send_set(GPIO_KEY, GPIO_PIN & GPIO_MASK);
  return 0;
}

/* unused */
uint8_t bano_get_handler(uint16_t key, uint32_t* val) { return (uint8_t)-1; }
uint8_t bano_set_handler(uint16_t key, uint32_t val) { return (uint8_t)-1; }
uint8_t bano_timer_handler(void) { return (uint8_t)-1; }

int main(void)
{
  bano_info_t info;

  info = bano_info_default;

  info.wake_mask |= BANO_WAKE_PCINT;
  info.pcint_mask = 1UL << 19UL;
  bano_init(&info);

  bano_loop();

  bano_fini();

  return 0;
}
\end{verbatim}
\end{tiny}


\end{document}
