\documentclass[a4paper, 11pt]{article}

\usepackage[margin=1in]{geometry}

\usepackage{hyperref}
\usepackage{graphicx}
\usepackage{graphics}
\usepackage{verbatim}
\usepackage{listings}
\usepackage{color}

\begin{document}

\title{BANO node API}
\author{texane@gmail.com}
\date{}

\maketitle

%% \newpage
%% \tableofcontents
%% \addtocontents{toc}{\protect\setcounter{tocdepth}{1}}


\clearpage
\section{Overview}

\paragraph{}
This document describes the BANO node application programming interface.
It is intended for node developers.

\paragraph{}
BANO offers a runtime and the corresponding programming interface that
abstracts a node application logic from low level details such as hardware
architecture and protocol implementation. The interface is mainly descriptive
and event based: the developer first initializes node related information.
The BANO runtime then calls application handlers whenever appropriate:
network messages reception, timers, hardware related interrupts ...


\clearpage
\section{Reference}

\subsection{Files}
common/bano\_common.h: constants and types common to base and node\\
node/bano\_node.h: function and type declarations\\
node/bano\_node.c: function implementations

\subsection{Types}
\begin{scriptsize}
\begin{verbatim}
typedef struct
{
  /* 100 milliseconds units, max 10736 */
  uint16_t timer_100ms;

  /* waking event mask */
#define BANO_WAKE_NONE 0
#define BANO_WAKE_TIMER (1 << 0)
#define BANO_WAKE_MSG (1 << 1)
#define BANO_WAKE_POLL (1 << 2)
#define BANO_WAKE_PCINT (1 << 3)
  uint8_t wake_mask;

  /* module disabling mask */
#define BANO_DISABLE_ADC (1 << 0)
#define BANO_DISABLE_WDT (1 << 1)
#define BANO_DISABLE_CMP (1 << 2)
#define BANO_DISABLE_USART (1 << 3)
#define BANO_DISABLE_NONE 0x00
#define BANO_DISABLE_ALL 0xff
  uint8_t disable_mask;

  uint32_t pcint_mask;

  /* NODL identifier */
  uint32_t nodl_id;

} bano_info_t;

static const bano_info_t bano_info_default =
{
  .wake_mask = BANO_WAKE_NONE,
  .disable_mask = BANO_DISABLE_ALL,
  .nodl_id = 0
};
\end{verbatim}
\end{scriptsize}


\subsection{Functions}
\begin{scriptsize}
\begin{verbatim}
/* exported by the runtime */
uint8_t bano_init(const bano_info_t*);
uint8_t bano_fini(void);
uint8_t bano_send_set(uint16_t, uint32_t);
uint8_t bano_wait_event(bano_msg_t*);
uint8_t bano_loop(void);
\end{verbatim}
\end{scriptsize}

\begin{scriptsize}
\begin{verbatim}
/* implemented by the application */
extern uint8_t bano_set_handler(uint16_t, uint32_t);
extern uint8_t bano_get_handler(uint16_t, uint32_t*);
extern uint8_t bano_timer_handler(void);
extern uint8_t bano_pcint_handler(void);
\end{verbatim}
\end{scriptsize}

\clearpage
\section{Example}

\subsection{Enable disable a LED on BANO messages}
\begin{tiny}
\begin{verbatim}
#include "bano/src/common/bano_common.h"
#include "bano/src/node/bano_node.h"
#include "bano/src/node/bano_node.c"


/* led routines */

#define LED_DDR DDRB
#define LED_PORT PORTB
#define LED_MASK (1 << 1)

static void led_set_high(void)
{
  LED_DDR |= LED_MASK;
  LED_PORT |= LED_MASK;
}

static void led_set_low(void)
{
  LED_DDR |= LED_MASK;
  LED_PORT &= ~LED_MASK;
}


/* event handlers */

#define LED_KEY 0x0000

static uint8_t led_value = 0;

uint8_t bano_get_handler(uint16_t key, uint32_t* val)
{
  /* called by the runtime on GET messages */

  if (key != LED_KEY) return (uint8_t)-1;
  *val = led_value;
  return 0;
}

uint8_t bano_set_handler(uint16_t key, uint32_t val)
{
  /* called by the runtime on SET messages */

  if (key != LED_KEY) return (uint8_t)-1;
  led_value = val;
  if (led_value == 0) led_set_low();
  else led_set_high();
  return 0;
}

int main(void)
{
  /* initialize the runtime and loop forever */

  bano_info_t info;

  info = bano_info_default;
  info.wake_mask |= BANO_WAKE_MSG;
  info.nodl_id = 0xdeadbeef;
  bano_init(&info);

  bano_loop();

  bano_fini();

  return 0;
}

\end{verbatim}
\end{tiny}


\clearpage
\subsection{Send periodic messages}
\begin{tiny}
\begin{verbatim}
#include "bano/src/common/bano_common.h"
#include "bano/src/node/bano_node.h"
#include "bano/src/node/bano_node.c"

uint8_t bano_timer_handler(void)
{
  /* called every 10 seconds */

  bano_send_set(0x002a, 0xdeadbeef);
  return 0;
}

int main(void)
{
  bano_info_t info;

  info = bano_info_default;

  info.wake_mask |= BANO_WAKE_TIMER;
  info.timer_100ms = 100;

  info.nodl_id = 0xdeadbeef;
  bano_init(&info);

  bano_loop();

  bano_fini();

  return 0;
}
\end{verbatim}
\end{tiny}


\clearpage
\subsection{Send message when GPIO changes}
\begin{tiny}
\begin{verbatim}
#include "bano/src/common/bano_common.h"
#include "bano/src/node/bano_node.h"
#include "bano/src/node/bano_node.c"

#define GPIO_KEY 0x0000

#define GPIO_DDR DDRD
#define GPIO_PIN PIND
#define GPIO_MASK (1 << 3)

uint8_t bano_pcint_handler(void)
{
  bano_send_set(GPIO_KEY, GPIO_PIN & GPIO_MASK);
  return 0;
}

int main(void)
{
  bano_info_t info;

  info = bano_info_default;

  info.wake_mask |= BANO_WAKE_PCINT;
  info.pcint_mask = 1UL << 19UL;
  info.nodl_id = 0xdeadbeef;
  bano_init(&info);

  bano_loop();

  bano_fini();

  return 0;
}
\end{verbatim}
\end{tiny}


\end{document}
